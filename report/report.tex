\documentclass{report}

\usepackage[a4paper]{geometry}
\usepackage{hyperref}
\usepackage{xcolor}
\usepackage{listings}
\lstdefinelanguage{SPL}{%
	alsoletter={0123456789ABCDEFGHIJKLMNOPQRSTUVWXYZabcdefghijklmnopqrstuvwxyz_+-*/\%=<>!\&|}
	morekeywords={while,else,if,print,return,isEmpty,fst,snd,hd,tl,Int,Char,Bool,Void,True,False,var},%
	sensitive=true,%
	morecomment=[l]{//},%
	morecomment=[n]{/*}{*/},%
	literate=%
		{->}{{$\rightarrow$}}2
		{>=}{{$\geq\:$}}1
		{<=}{{$\leq\:$}}1
		{==}{{$\equiv$}}1
}
\lstset{%
	upquote=true,
	breakatwhitespace=false,
	breaklines=true,
	postbreak=\mbox{\textcolor{gray}{$\hookrightarrow$}\space},
	keepspaces=true,
	basicstyle=\tt\footnotesize,
	commentstyle=\sl,
	keywordstyle=\bf,
	stringstyle=\tt,
	showspaces=false,
	showstringspaces=false,
	showtabs=false,
	tabsize=4,
	basewidth=0.43em,
	columns=[c]fixed,
	texcl=true,
	captionpos=b
}

\author{%
	Sjaak~Smetsers\\
	\small\texttt{s123456}\and
	Mart~Lubbers\\
	\small\texttt{s123456}
}
\date{\today}
\title{My compiler}

\begin{document}

\maketitle%

\tableofcontents%

\chapter{Introduction}
Motivate your language choice, introduce spl.

\begin{lstlisting}[language=SPL,numbers=left]
foo (n) :: Int -> (Int, Int)
{
	return (2, 2);
}

transpose (p1, p2) :: (Int, Int) (Int, Int) -> (Int, Int)
{
	return ((p1.fst + p2.fst), (p1.snd + p2.snd));
}

scale(p, scalar) :: (Int, Int) Int -> (Int, Int) {
	return (p.fst * scalar, p.snd * scalar);
}
\end{lstlisting}

\chapter{Lexing \& Parsing}
\begin{itemize}
	\item How did you design the Abstract Syntax Tree
	\item How does the parser work?
	\item Is there error handling? Recovery?
	\item Do you have a lexer and parser?
	\item How do they communicate?
	\item Problems?
	\item\ldots
\end{itemize}

\chapter{Analyses \& Typing}
\begin{itemize}
	\item New Abstract Syntax Tree? Decorate existing Abstract Syntax Tree?
	\item Error messages?
	\item Polymorphism? Inference? Overloading?
	\item Problems?
	\item\ldots
\end{itemize}

\chapter{Code Generation}
\begin{itemize}
	\item Compilation scheme?
	\item How is data represented? Lists tuples
	\item Semantics style, call-by-reference, call-by-value?
	\item How did you solve overloaded functions?
	\item Polymorphism?
	\item Printing?
	\item Problems?
	\item\ldots
\end{itemize}

\chapter{Extension}
Describe your extension in detail

\chapter{Conclusion}
What does work, what does not etc.

\section{Reflection}
\begin{itemize}
	\item What do you think of the project?
	\item How did it work out?
	\item How did you divide the work?
	\item Pitfalls?
	\item \ldots
\end{itemize}

\appendix
\chapter{Grammar}
Change the grammar to the one you actually used

\begin{verbatim}
SPL       = Decl+
Decl      = VarDecl
          | FunDecl
VarDecl   = ('var' | Type) id  '=' Exp ';'
FunDecl   = id '(' [ FArgs ] ')' [ '::' FunType ] '{' VarDecl* Stmt+ '}'
RetType   = Type
          | 'Void'
FunType   = [ FTypes ] '->' RetType
FTypes    = Type [ FTypes ]
Type      = BasicType
          | '(' Type ',' Type ')'
          | '[' Type ']'
          | id
BasicType = 'Int'
          | 'Bool'
          | 'Char'
FArgs     = [ FArgs ',' ] id
Stmt      = 'if' '(' Exp ')' '{' Stmt* '}' [ 'else' '{' Stmt* '}' ]
          | 'while' '(' Exp ')' '{' Stmt* '}'
          | id Field '=' Exp ';'
          | FunCall ';'
          | 'return' [ Exp ] ';'
Exp       = id Field
          | Exp Op2 Exp
          | Op1 Exp
          | int
          | char
          | 'False' | 'True'
          | '(' Exp ')'
          | FunCall
          | '[]'
          | '(' Exp ',' Exp ')'
Field     = [ Field ( '.' 'hd' | '.' 'tl' | '.' 'fst' | '.' 'snd' ) ]
FunCall   = id '(' [ ActArgs ] ')'
ActArgs   = Exp [ ',' ActArgs ]
Op2       = '+'  | '-' | '*' | '/'  | '%'
          | '==' | '<' | '>' | '<=' | '>=' | '!='
          | '&&' | '||'
          | ':'
Op1       = '!'  | '-'
int       = [ '-' ] digit+
id        = alpha ( '_' | alphaNum)*
\end{verbatim}

\end{document}
